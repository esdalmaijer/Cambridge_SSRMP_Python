
% Default to the notebook output style

    


% Inherit from the specified cell style.




    
\documentclass[11pt]{article}

    
    
    \usepackage[T1]{fontenc}
    % Nicer default font (+ math font) than Computer Modern for most use cases
    \usepackage{mathpazo}

    % Basic figure setup, for now with no caption control since it's done
    % automatically by Pandoc (which extracts ![](path) syntax from Markdown).
    \usepackage{graphicx}
    % We will generate all images so they have a width \maxwidth. This means
    % that they will get their normal width if they fit onto the page, but
    % are scaled down if they would overflow the margins.
    \makeatletter
    \def\maxwidth{\ifdim\Gin@nat@width>\linewidth\linewidth
    \else\Gin@nat@width\fi}
    \makeatother
    \let\Oldincludegraphics\includegraphics
    % Set max figure width to be 80% of text width, for now hardcoded.
    \renewcommand{\includegraphics}[1]{\Oldincludegraphics[width=.8\maxwidth]{#1}}
    % Ensure that by default, figures have no caption (until we provide a
    % proper Figure object with a Caption API and a way to capture that
    % in the conversion process - todo).
    \usepackage{caption}
    \DeclareCaptionLabelFormat{nolabel}{}
    \captionsetup{labelformat=nolabel}

    \usepackage{adjustbox} % Used to constrain images to a maximum size 
    \usepackage{xcolor} % Allow colors to be defined
    \usepackage{enumerate} % Needed for markdown enumerations to work
    \usepackage{geometry} % Used to adjust the document margins
    \usepackage{amsmath} % Equations
    \usepackage{amssymb} % Equations
    \usepackage{textcomp} % defines textquotesingle
    % Hack from http://tex.stackexchange.com/a/47451/13684:
    \AtBeginDocument{%
        \def\PYZsq{\textquotesingle}% Upright quotes in Pygmentized code
    }
    \usepackage{upquote} % Upright quotes for verbatim code
    \usepackage{eurosym} % defines \euro
    \usepackage[mathletters]{ucs} % Extended unicode (utf-8) support
    \usepackage[utf8x]{inputenc} % Allow utf-8 characters in the tex document
    \usepackage{fancyvrb} % verbatim replacement that allows latex
    \usepackage{grffile} % extends the file name processing of package graphics 
                         % to support a larger range 
    % The hyperref package gives us a pdf with properly built
    % internal navigation ('pdf bookmarks' for the table of contents,
    % internal cross-reference links, web links for URLs, etc.)
    \usepackage{hyperref}
    \usepackage{longtable} % longtable support required by pandoc >1.10
    \usepackage{booktabs}  % table support for pandoc > 1.12.2
    \usepackage[inline]{enumitem} % IRkernel/repr support (it uses the enumerate* environment)
    \usepackage[normalem]{ulem} % ulem is needed to support strikethroughs (\sout)
                                % normalem makes italics be italics, not underlines
    

    
    
    % Colors for the hyperref package
    \definecolor{urlcolor}{rgb}{0,.145,.698}
    \definecolor{linkcolor}{rgb}{.71,0.21,0.01}
    \definecolor{citecolor}{rgb}{.12,.54,.11}

    % ANSI colors
    \definecolor{ansi-black}{HTML}{3E424D}
    \definecolor{ansi-black-intense}{HTML}{282C36}
    \definecolor{ansi-red}{HTML}{E75C58}
    \definecolor{ansi-red-intense}{HTML}{B22B31}
    \definecolor{ansi-green}{HTML}{00A250}
    \definecolor{ansi-green-intense}{HTML}{007427}
    \definecolor{ansi-yellow}{HTML}{DDB62B}
    \definecolor{ansi-yellow-intense}{HTML}{B27D12}
    \definecolor{ansi-blue}{HTML}{208FFB}
    \definecolor{ansi-blue-intense}{HTML}{0065CA}
    \definecolor{ansi-magenta}{HTML}{D160C4}
    \definecolor{ansi-magenta-intense}{HTML}{A03196}
    \definecolor{ansi-cyan}{HTML}{60C6C8}
    \definecolor{ansi-cyan-intense}{HTML}{258F8F}
    \definecolor{ansi-white}{HTML}{C5C1B4}
    \definecolor{ansi-white-intense}{HTML}{A1A6B2}

    % commands and environments needed by pandoc snippets
    % extracted from the output of `pandoc -s`
    \providecommand{\tightlist}{%
      \setlength{\itemsep}{0pt}\setlength{\parskip}{0pt}}
    \DefineVerbatimEnvironment{Highlighting}{Verbatim}{commandchars=\\\{\}}
    % Add ',fontsize=\small' for more characters per line
    \newenvironment{Shaded}{}{}
    \newcommand{\KeywordTok}[1]{\textcolor[rgb]{0.00,0.44,0.13}{\textbf{{#1}}}}
    \newcommand{\DataTypeTok}[1]{\textcolor[rgb]{0.56,0.13,0.00}{{#1}}}
    \newcommand{\DecValTok}[1]{\textcolor[rgb]{0.25,0.63,0.44}{{#1}}}
    \newcommand{\BaseNTok}[1]{\textcolor[rgb]{0.25,0.63,0.44}{{#1}}}
    \newcommand{\FloatTok}[1]{\textcolor[rgb]{0.25,0.63,0.44}{{#1}}}
    \newcommand{\CharTok}[1]{\textcolor[rgb]{0.25,0.44,0.63}{{#1}}}
    \newcommand{\StringTok}[1]{\textcolor[rgb]{0.25,0.44,0.63}{{#1}}}
    \newcommand{\CommentTok}[1]{\textcolor[rgb]{0.38,0.63,0.69}{\textit{{#1}}}}
    \newcommand{\OtherTok}[1]{\textcolor[rgb]{0.00,0.44,0.13}{{#1}}}
    \newcommand{\AlertTok}[1]{\textcolor[rgb]{1.00,0.00,0.00}{\textbf{{#1}}}}
    \newcommand{\FunctionTok}[1]{\textcolor[rgb]{0.02,0.16,0.49}{{#1}}}
    \newcommand{\RegionMarkerTok}[1]{{#1}}
    \newcommand{\ErrorTok}[1]{\textcolor[rgb]{1.00,0.00,0.00}{\textbf{{#1}}}}
    \newcommand{\NormalTok}[1]{{#1}}
    
    % Additional commands for more recent versions of Pandoc
    \newcommand{\ConstantTok}[1]{\textcolor[rgb]{0.53,0.00,0.00}{{#1}}}
    \newcommand{\SpecialCharTok}[1]{\textcolor[rgb]{0.25,0.44,0.63}{{#1}}}
    \newcommand{\VerbatimStringTok}[1]{\textcolor[rgb]{0.25,0.44,0.63}{{#1}}}
    \newcommand{\SpecialStringTok}[1]{\textcolor[rgb]{0.73,0.40,0.53}{{#1}}}
    \newcommand{\ImportTok}[1]{{#1}}
    \newcommand{\DocumentationTok}[1]{\textcolor[rgb]{0.73,0.13,0.13}{\textit{{#1}}}}
    \newcommand{\AnnotationTok}[1]{\textcolor[rgb]{0.38,0.63,0.69}{\textbf{\textit{{#1}}}}}
    \newcommand{\CommentVarTok}[1]{\textcolor[rgb]{0.38,0.63,0.69}{\textbf{\textit{{#1}}}}}
    \newcommand{\VariableTok}[1]{\textcolor[rgb]{0.10,0.09,0.49}{{#1}}}
    \newcommand{\ControlFlowTok}[1]{\textcolor[rgb]{0.00,0.44,0.13}{\textbf{{#1}}}}
    \newcommand{\OperatorTok}[1]{\textcolor[rgb]{0.40,0.40,0.40}{{#1}}}
    \newcommand{\BuiltInTok}[1]{{#1}}
    \newcommand{\ExtensionTok}[1]{{#1}}
    \newcommand{\PreprocessorTok}[1]{\textcolor[rgb]{0.74,0.48,0.00}{{#1}}}
    \newcommand{\AttributeTok}[1]{\textcolor[rgb]{0.49,0.56,0.16}{{#1}}}
    \newcommand{\InformationTok}[1]{\textcolor[rgb]{0.38,0.63,0.69}{\textbf{\textit{{#1}}}}}
    \newcommand{\WarningTok}[1]{\textcolor[rgb]{0.38,0.63,0.69}{\textbf{\textit{{#1}}}}}
    
    
    % Define a nice break command that doesn't care if a line doesn't already
    % exist.
    \def\br{\hspace*{\fill} \\* }
    % Math Jax compatability definitions
    \def\gt{>}
    \def\lt{<}
    % Document parameters
    \title{python\_basics}
    
    
    

    % Pygments definitions
    
\makeatletter
\def\PY@reset{\let\PY@it=\relax \let\PY@bf=\relax%
    \let\PY@ul=\relax \let\PY@tc=\relax%
    \let\PY@bc=\relax \let\PY@ff=\relax}
\def\PY@tok#1{\csname PY@tok@#1\endcsname}
\def\PY@toks#1+{\ifx\relax#1\empty\else%
    \PY@tok{#1}\expandafter\PY@toks\fi}
\def\PY@do#1{\PY@bc{\PY@tc{\PY@ul{%
    \PY@it{\PY@bf{\PY@ff{#1}}}}}}}
\def\PY#1#2{\PY@reset\PY@toks#1+\relax+\PY@do{#2}}

\expandafter\def\csname PY@tok@gd\endcsname{\def\PY@tc##1{\textcolor[rgb]{0.63,0.00,0.00}{##1}}}
\expandafter\def\csname PY@tok@gu\endcsname{\let\PY@bf=\textbf\def\PY@tc##1{\textcolor[rgb]{0.50,0.00,0.50}{##1}}}
\expandafter\def\csname PY@tok@gt\endcsname{\def\PY@tc##1{\textcolor[rgb]{0.00,0.27,0.87}{##1}}}
\expandafter\def\csname PY@tok@gs\endcsname{\let\PY@bf=\textbf}
\expandafter\def\csname PY@tok@gr\endcsname{\def\PY@tc##1{\textcolor[rgb]{1.00,0.00,0.00}{##1}}}
\expandafter\def\csname PY@tok@cm\endcsname{\let\PY@it=\textit\def\PY@tc##1{\textcolor[rgb]{0.25,0.50,0.50}{##1}}}
\expandafter\def\csname PY@tok@vg\endcsname{\def\PY@tc##1{\textcolor[rgb]{0.10,0.09,0.49}{##1}}}
\expandafter\def\csname PY@tok@vi\endcsname{\def\PY@tc##1{\textcolor[rgb]{0.10,0.09,0.49}{##1}}}
\expandafter\def\csname PY@tok@vm\endcsname{\def\PY@tc##1{\textcolor[rgb]{0.10,0.09,0.49}{##1}}}
\expandafter\def\csname PY@tok@mh\endcsname{\def\PY@tc##1{\textcolor[rgb]{0.40,0.40,0.40}{##1}}}
\expandafter\def\csname PY@tok@cs\endcsname{\let\PY@it=\textit\def\PY@tc##1{\textcolor[rgb]{0.25,0.50,0.50}{##1}}}
\expandafter\def\csname PY@tok@ge\endcsname{\let\PY@it=\textit}
\expandafter\def\csname PY@tok@vc\endcsname{\def\PY@tc##1{\textcolor[rgb]{0.10,0.09,0.49}{##1}}}
\expandafter\def\csname PY@tok@il\endcsname{\def\PY@tc##1{\textcolor[rgb]{0.40,0.40,0.40}{##1}}}
\expandafter\def\csname PY@tok@go\endcsname{\def\PY@tc##1{\textcolor[rgb]{0.53,0.53,0.53}{##1}}}
\expandafter\def\csname PY@tok@cp\endcsname{\def\PY@tc##1{\textcolor[rgb]{0.74,0.48,0.00}{##1}}}
\expandafter\def\csname PY@tok@gi\endcsname{\def\PY@tc##1{\textcolor[rgb]{0.00,0.63,0.00}{##1}}}
\expandafter\def\csname PY@tok@gh\endcsname{\let\PY@bf=\textbf\def\PY@tc##1{\textcolor[rgb]{0.00,0.00,0.50}{##1}}}
\expandafter\def\csname PY@tok@ni\endcsname{\let\PY@bf=\textbf\def\PY@tc##1{\textcolor[rgb]{0.60,0.60,0.60}{##1}}}
\expandafter\def\csname PY@tok@nl\endcsname{\def\PY@tc##1{\textcolor[rgb]{0.63,0.63,0.00}{##1}}}
\expandafter\def\csname PY@tok@nn\endcsname{\let\PY@bf=\textbf\def\PY@tc##1{\textcolor[rgb]{0.00,0.00,1.00}{##1}}}
\expandafter\def\csname PY@tok@no\endcsname{\def\PY@tc##1{\textcolor[rgb]{0.53,0.00,0.00}{##1}}}
\expandafter\def\csname PY@tok@na\endcsname{\def\PY@tc##1{\textcolor[rgb]{0.49,0.56,0.16}{##1}}}
\expandafter\def\csname PY@tok@nb\endcsname{\def\PY@tc##1{\textcolor[rgb]{0.00,0.50,0.00}{##1}}}
\expandafter\def\csname PY@tok@nc\endcsname{\let\PY@bf=\textbf\def\PY@tc##1{\textcolor[rgb]{0.00,0.00,1.00}{##1}}}
\expandafter\def\csname PY@tok@nd\endcsname{\def\PY@tc##1{\textcolor[rgb]{0.67,0.13,1.00}{##1}}}
\expandafter\def\csname PY@tok@ne\endcsname{\let\PY@bf=\textbf\def\PY@tc##1{\textcolor[rgb]{0.82,0.25,0.23}{##1}}}
\expandafter\def\csname PY@tok@nf\endcsname{\def\PY@tc##1{\textcolor[rgb]{0.00,0.00,1.00}{##1}}}
\expandafter\def\csname PY@tok@si\endcsname{\let\PY@bf=\textbf\def\PY@tc##1{\textcolor[rgb]{0.73,0.40,0.53}{##1}}}
\expandafter\def\csname PY@tok@s2\endcsname{\def\PY@tc##1{\textcolor[rgb]{0.73,0.13,0.13}{##1}}}
\expandafter\def\csname PY@tok@nt\endcsname{\let\PY@bf=\textbf\def\PY@tc##1{\textcolor[rgb]{0.00,0.50,0.00}{##1}}}
\expandafter\def\csname PY@tok@nv\endcsname{\def\PY@tc##1{\textcolor[rgb]{0.10,0.09,0.49}{##1}}}
\expandafter\def\csname PY@tok@s1\endcsname{\def\PY@tc##1{\textcolor[rgb]{0.73,0.13,0.13}{##1}}}
\expandafter\def\csname PY@tok@dl\endcsname{\def\PY@tc##1{\textcolor[rgb]{0.73,0.13,0.13}{##1}}}
\expandafter\def\csname PY@tok@ch\endcsname{\let\PY@it=\textit\def\PY@tc##1{\textcolor[rgb]{0.25,0.50,0.50}{##1}}}
\expandafter\def\csname PY@tok@m\endcsname{\def\PY@tc##1{\textcolor[rgb]{0.40,0.40,0.40}{##1}}}
\expandafter\def\csname PY@tok@gp\endcsname{\let\PY@bf=\textbf\def\PY@tc##1{\textcolor[rgb]{0.00,0.00,0.50}{##1}}}
\expandafter\def\csname PY@tok@sh\endcsname{\def\PY@tc##1{\textcolor[rgb]{0.73,0.13,0.13}{##1}}}
\expandafter\def\csname PY@tok@ow\endcsname{\let\PY@bf=\textbf\def\PY@tc##1{\textcolor[rgb]{0.67,0.13,1.00}{##1}}}
\expandafter\def\csname PY@tok@sx\endcsname{\def\PY@tc##1{\textcolor[rgb]{0.00,0.50,0.00}{##1}}}
\expandafter\def\csname PY@tok@bp\endcsname{\def\PY@tc##1{\textcolor[rgb]{0.00,0.50,0.00}{##1}}}
\expandafter\def\csname PY@tok@c1\endcsname{\let\PY@it=\textit\def\PY@tc##1{\textcolor[rgb]{0.25,0.50,0.50}{##1}}}
\expandafter\def\csname PY@tok@fm\endcsname{\def\PY@tc##1{\textcolor[rgb]{0.00,0.00,1.00}{##1}}}
\expandafter\def\csname PY@tok@o\endcsname{\def\PY@tc##1{\textcolor[rgb]{0.40,0.40,0.40}{##1}}}
\expandafter\def\csname PY@tok@kc\endcsname{\let\PY@bf=\textbf\def\PY@tc##1{\textcolor[rgb]{0.00,0.50,0.00}{##1}}}
\expandafter\def\csname PY@tok@c\endcsname{\let\PY@it=\textit\def\PY@tc##1{\textcolor[rgb]{0.25,0.50,0.50}{##1}}}
\expandafter\def\csname PY@tok@mf\endcsname{\def\PY@tc##1{\textcolor[rgb]{0.40,0.40,0.40}{##1}}}
\expandafter\def\csname PY@tok@err\endcsname{\def\PY@bc##1{\setlength{\fboxsep}{0pt}\fcolorbox[rgb]{1.00,0.00,0.00}{1,1,1}{\strut ##1}}}
\expandafter\def\csname PY@tok@mb\endcsname{\def\PY@tc##1{\textcolor[rgb]{0.40,0.40,0.40}{##1}}}
\expandafter\def\csname PY@tok@ss\endcsname{\def\PY@tc##1{\textcolor[rgb]{0.10,0.09,0.49}{##1}}}
\expandafter\def\csname PY@tok@sr\endcsname{\def\PY@tc##1{\textcolor[rgb]{0.73,0.40,0.53}{##1}}}
\expandafter\def\csname PY@tok@mo\endcsname{\def\PY@tc##1{\textcolor[rgb]{0.40,0.40,0.40}{##1}}}
\expandafter\def\csname PY@tok@kd\endcsname{\let\PY@bf=\textbf\def\PY@tc##1{\textcolor[rgb]{0.00,0.50,0.00}{##1}}}
\expandafter\def\csname PY@tok@mi\endcsname{\def\PY@tc##1{\textcolor[rgb]{0.40,0.40,0.40}{##1}}}
\expandafter\def\csname PY@tok@kn\endcsname{\let\PY@bf=\textbf\def\PY@tc##1{\textcolor[rgb]{0.00,0.50,0.00}{##1}}}
\expandafter\def\csname PY@tok@cpf\endcsname{\let\PY@it=\textit\def\PY@tc##1{\textcolor[rgb]{0.25,0.50,0.50}{##1}}}
\expandafter\def\csname PY@tok@kr\endcsname{\let\PY@bf=\textbf\def\PY@tc##1{\textcolor[rgb]{0.00,0.50,0.00}{##1}}}
\expandafter\def\csname PY@tok@s\endcsname{\def\PY@tc##1{\textcolor[rgb]{0.73,0.13,0.13}{##1}}}
\expandafter\def\csname PY@tok@kp\endcsname{\def\PY@tc##1{\textcolor[rgb]{0.00,0.50,0.00}{##1}}}
\expandafter\def\csname PY@tok@w\endcsname{\def\PY@tc##1{\textcolor[rgb]{0.73,0.73,0.73}{##1}}}
\expandafter\def\csname PY@tok@kt\endcsname{\def\PY@tc##1{\textcolor[rgb]{0.69,0.00,0.25}{##1}}}
\expandafter\def\csname PY@tok@sc\endcsname{\def\PY@tc##1{\textcolor[rgb]{0.73,0.13,0.13}{##1}}}
\expandafter\def\csname PY@tok@sb\endcsname{\def\PY@tc##1{\textcolor[rgb]{0.73,0.13,0.13}{##1}}}
\expandafter\def\csname PY@tok@sa\endcsname{\def\PY@tc##1{\textcolor[rgb]{0.73,0.13,0.13}{##1}}}
\expandafter\def\csname PY@tok@k\endcsname{\let\PY@bf=\textbf\def\PY@tc##1{\textcolor[rgb]{0.00,0.50,0.00}{##1}}}
\expandafter\def\csname PY@tok@se\endcsname{\let\PY@bf=\textbf\def\PY@tc##1{\textcolor[rgb]{0.73,0.40,0.13}{##1}}}
\expandafter\def\csname PY@tok@sd\endcsname{\let\PY@it=\textit\def\PY@tc##1{\textcolor[rgb]{0.73,0.13,0.13}{##1}}}

\def\PYZbs{\char`\\}
\def\PYZus{\char`\_}
\def\PYZob{\char`\{}
\def\PYZcb{\char`\}}
\def\PYZca{\char`\^}
\def\PYZam{\char`\&}
\def\PYZlt{\char`\<}
\def\PYZgt{\char`\>}
\def\PYZsh{\char`\#}
\def\PYZpc{\char`\%}
\def\PYZdl{\char`\$}
\def\PYZhy{\char`\-}
\def\PYZsq{\char`\'}
\def\PYZdq{\char`\"}
\def\PYZti{\char`\~}
% for compatibility with earlier versions
\def\PYZat{@}
\def\PYZlb{[}
\def\PYZrb{]}
\makeatother


    % Exact colors from NB
    \definecolor{incolor}{rgb}{0.0, 0.0, 0.5}
    \definecolor{outcolor}{rgb}{0.545, 0.0, 0.0}



    
    % Prevent overflowing lines due to hard-to-break entities
    \sloppy 
    % Setup hyperref package
    \hypersetup{
      breaklinks=true,  % so long urls are correctly broken across lines
      colorlinks=true,
      urlcolor=urlcolor,
      linkcolor=linkcolor,
      citecolor=citecolor,
      }
    % Slightly bigger margins than the latex defaults
    
    \geometry{verbose,tmargin=1in,bmargin=1in,lmargin=1in,rmargin=1in}
    
    

    \begin{document}
    
    
    \maketitle
    
    

    
    \section{Python Basics}\label{python-basics}

This worksheet will cover the basics of Python: types of values,
variables, statements, functions, and classes. First, let's get some
jargon out of the way.

\begin{itemize}
\tightlist
\item
  \textbf{value type}: Programming languages can recognise several
  specific value types, including numbers and text. We'll go through all
  important types in this worksheet.
\item
  \textbf{variable}: A variable has a name, and is linked to a value.
  It's a way of keeping track of specific values in memory. For example,
  you could say \texttt{a\ =\ 3}. The value \texttt{3} is now linked to
  the variable \texttt{a}. You can refer to \texttt{a} later in your
  script.
\item
  \textbf{statement}: There are several types of statements. They are
  the building blocks of a programming language, and allow you to
  repeatedly run small pockets of code, or to run a bit of code only if
  a certain condition is met.
\item
  \textbf{function}: A function is a defined collection of code. It
  usually takes some input, processes this, and then returns some
  output. For example, the function \texttt{sum} takes as input a
  collection of numbers, adds them all together, and then returns the
  sum. Functions can be short and simple, or long and complex, and
  anything in between. Some functions are built into Python, some are
  inlcuded in external packages, and you can also write your own.
\item
  \textbf{class}: A class is a collection of variables and functions,
  all linked to the same \textbf{instance}. For example, you can think
  of a car as an \emph{instance}: It has its own properties (like its
  top speed), and its own functions (like driving). The blueprint for a
  car is more like a \emph{class}: It describes how a car should be
  built, for example with four wheels. Even when produced with the same
  blueprint (class), cars (instances) can be different from each other.
  For example, they could have different wheels, or a different colour.
\end{itemize}

If these descriptions seem a bit unclear still, don't worry: We're going
to go through them in more detail in this worksheet.

    \subsection{Numbers}\label{numbers}

Numbers come in two formats: integers and floats. \textbf{Integers} are
whole, undivided numbers like -2, 0, and 15. \textbf{Floats} (floating
point) numbers are fractions, for example 3.1, 5.43, and
0.6666666666666667.

    \subsubsection{Integers}\label{integers}

You can use integers in basic operations like addition, subtraction,
multiplication, and division:

    \begin{Verbatim}[commandchars=\\\{\}]
{\color{incolor}In [{\color{incolor}1}]:} \PY{l+m+mi}{1} \PY{o}{+} \PY{l+m+mi}{1}
\end{Verbatim}


\begin{Verbatim}[commandchars=\\\{\}]
{\color{outcolor}Out[{\color{outcolor}1}]:} 2
\end{Verbatim}
            
    \begin{Verbatim}[commandchars=\\\{\}]
{\color{incolor}In [{\color{incolor}2}]:} \PY{l+m+mi}{5} \PY{o}{\PYZhy{}} \PY{l+m+mi}{3}
\end{Verbatim}


\begin{Verbatim}[commandchars=\\\{\}]
{\color{outcolor}Out[{\color{outcolor}2}]:} 2
\end{Verbatim}
            
    \begin{Verbatim}[commandchars=\\\{\}]
{\color{incolor}In [{\color{incolor}3}]:} \PY{l+m+mi}{10} \PY{o}{\PYZhy{}} \PY{l+m+mi}{900}
\end{Verbatim}


\begin{Verbatim}[commandchars=\\\{\}]
{\color{outcolor}Out[{\color{outcolor}3}]:} -890
\end{Verbatim}
            
    \begin{Verbatim}[commandchars=\\\{\}]
{\color{incolor}In [{\color{incolor}4}]:} \PY{l+m+mi}{2} \PY{o}{*} \PY{l+m+mi}{10}
\end{Verbatim}


\begin{Verbatim}[commandchars=\\\{\}]
{\color{outcolor}Out[{\color{outcolor}4}]:} 20
\end{Verbatim}
            
    \begin{Verbatim}[commandchars=\\\{\}]
{\color{incolor}In [{\color{incolor}5}]:} \PY{l+m+mi}{10} \PY{o}{/} \PY{l+m+mi}{5}
\end{Verbatim}


\begin{Verbatim}[commandchars=\\\{\}]
{\color{outcolor}Out[{\color{outcolor}5}]:} 2
\end{Verbatim}
            
    Let's try allocating an integer to a \textbf{variable}, and do
operations with that:

    \begin{Verbatim}[commandchars=\\\{\}]
{\color{incolor}In [{\color{incolor}6}]:} \PY{n}{a} \PY{o}{=} \PY{l+m+mi}{3}
        \PY{n}{a} \PY{o}{*} \PY{l+m+mi}{2}
\end{Verbatim}


\begin{Verbatim}[commandchars=\\\{\}]
{\color{outcolor}Out[{\color{outcolor}6}]:} 6
\end{Verbatim}
            
    \subsubsection{Floats}\label{floats}

Floats can also be used in basic operations:

    \begin{Verbatim}[commandchars=\\\{\}]
{\color{incolor}In [{\color{incolor}7}]:} \PY{l+m+mf}{0.5} \PY{o}{+} \PY{l+m+mf}{0.1}
\end{Verbatim}


\begin{Verbatim}[commandchars=\\\{\}]
{\color{outcolor}Out[{\color{outcolor}7}]:} 0.6
\end{Verbatim}
            
    \begin{Verbatim}[commandchars=\\\{\}]
{\color{incolor}In [{\color{incolor}8}]:} \PY{l+m+mf}{1.5} \PY{o}{\PYZhy{}} \PY{l+m+mf}{0.5}
\end{Verbatim}


\begin{Verbatim}[commandchars=\\\{\}]
{\color{outcolor}Out[{\color{outcolor}8}]:} 1.0
\end{Verbatim}
            
    \begin{Verbatim}[commandchars=\\\{\}]
{\color{incolor}In [{\color{incolor}9}]:} \PY{l+m+mf}{5.0} \PY{o}{*} \PY{l+m+mf}{0.25}
\end{Verbatim}


\begin{Verbatim}[commandchars=\\\{\}]
{\color{outcolor}Out[{\color{outcolor}9}]:} 1.25
\end{Verbatim}
            
    \begin{Verbatim}[commandchars=\\\{\}]
{\color{incolor}In [{\color{incolor}10}]:} \PY{l+m+mf}{1.25} \PY{o}{/} \PY{l+m+mf}{5.0}
\end{Verbatim}


\begin{Verbatim}[commandchars=\\\{\}]
{\color{outcolor}Out[{\color{outcolor}10}]:} 0.25
\end{Verbatim}
            
    You can do the same by assigning floats to variables:

    \begin{Verbatim}[commandchars=\\\{\}]
{\color{incolor}In [{\color{incolor}11}]:} \PY{n}{a} \PY{o}{=} \PY{l+m+mf}{0.5}
         \PY{n}{b} \PY{o}{=} \PY{l+m+mf}{3.0}
         \PY{n}{a} \PY{o}{*} \PY{n}{b}
\end{Verbatim}


\begin{Verbatim}[commandchars=\\\{\}]
{\color{outcolor}Out[{\color{outcolor}11}]:} 1.5
\end{Verbatim}
            
    \subsubsection{Operations}\label{operations}

Aside from the basic addition (\texttt{+}), subtraction (\texttt{-}),
multiplication (\texttt{*}), and division (\texttt{\textbackslash{}}),
there are also operators to do exponentiation (\texttt{**}), integer
division (\texttt{\textbackslash{}\textbackslash{}}), and to compute the
remainder after division (\texttt{\%}). Let's see them in action:

    \paragraph{Exponentiation}\label{exponentiation}

Three square (\(3^2\)) and two to the power of eight (\(2^8\)):

    \begin{Verbatim}[commandchars=\\\{\}]
{\color{incolor}In [{\color{incolor}12}]:} \PY{l+m+mi}{3}\PY{o}{*}\PY{o}{*}\PY{l+m+mi}{2}
\end{Verbatim}


\begin{Verbatim}[commandchars=\\\{\}]
{\color{outcolor}Out[{\color{outcolor}12}]:} 9
\end{Verbatim}
            
    \begin{Verbatim}[commandchars=\\\{\}]
{\color{incolor}In [{\color{incolor}13}]:} \PY{l+m+mi}{2}\PY{o}{*}\PY{o}{*}\PY{l+m+mi}{8}
\end{Verbatim}


\begin{Verbatim}[commandchars=\\\{\}]
{\color{outcolor}Out[{\color{outcolor}13}]:} 256
\end{Verbatim}
            
    \paragraph{Division (integer and
remainder)}\label{division-integer-and-remainder}

The result of a division can be a fraction, which can be expressed as a
float. However, sometimes it can be useful to know the result as an
integer and the remainder. For example, the outcome of five divided by
two is 2.5, but can also be expressed as 2 with a remainder of 1. In
Python, you can compute these as follows:

    \begin{Verbatim}[commandchars=\\\{\}]
{\color{incolor}In [{\color{incolor}14}]:} \PY{l+m+mi}{5} \PY{o}{/}\PY{o}{/} \PY{l+m+mi}{2}
\end{Verbatim}


\begin{Verbatim}[commandchars=\\\{\}]
{\color{outcolor}Out[{\color{outcolor}14}]:} 2
\end{Verbatim}
            
    \begin{Verbatim}[commandchars=\\\{\}]
{\color{incolor}In [{\color{incolor}15}]:} \PY{l+m+mi}{5} \PY{o}{\PYZpc{}} \PY{l+m+mi}{2}
\end{Verbatim}


\begin{Verbatim}[commandchars=\\\{\}]
{\color{outcolor}Out[{\color{outcolor}15}]:} 1
\end{Verbatim}
            
    \subsubsection{Should I be aware of the difference between int and
float?}\label{should-i-be-aware-of-the-difference-between-int-and-float}

Previously, divisions of two integers ignored the remainder. So
\texttt{5/2} would be the same as \texttt{5//2}, both resulting in
\texttt{2}. In Python 3 (the newest version), this is no longer the
case: \texttt{5/2} should result in \texttt{2.5}. While this is an
intuitive result to humans, many programming languages are more fussy
about variable types. That an operation involving two integers results
in a float could thus be considered a bit weird.

As a human, you might be relieved that Python's behaviour maps onto your
own, rather than on a more rigid interpretation of what numbers are.
There are people who argue that this means you don't have to play
attention to the distinction between integers and floats.

Those people are wrong. At some point, you'll encounter functions that
require integers, and can't deal with floats. This is important if
you're dealing with undivisible quantities, like pixels: there's no such
thing as half a pixel.

That means that while \texttt{5} and \texttt{3} are both OK pixel
values, \texttt{5/3} is NOT. This is easy to spot with numerical values,
but less so if you can only see variable names \texttt{a/b}.

    \subsection{Text strings}\label{text-strings}

Python can also deal with text values, in the form of the
\textbf{string} type. Strings are denoted by single
(\texttt{\textquotesingle{}}) or double (\texttt{"}) quotes, or even
three quotes
(\texttt{\textquotesingle{}\textquotesingle{}\textquotesingle{}} or
\texttt{"""}). Par example:

    \begin{Verbatim}[commandchars=\\\{\}]
{\color{incolor}In [{\color{incolor}16}]:} \PY{l+s+s2}{\PYZdq{}}\PY{l+s+s2}{This is a string}\PY{l+s+s2}{\PYZdq{}}
\end{Verbatim}


\begin{Verbatim}[commandchars=\\\{\}]
{\color{outcolor}Out[{\color{outcolor}16}]:} 'This is a string'
\end{Verbatim}
            
    \begin{Verbatim}[commandchars=\\\{\}]
{\color{incolor}In [{\color{incolor}17}]:} \PY{l+s+s1}{\PYZsq{}}\PY{l+s+s1}{This is also a string}\PY{l+s+s1}{\PYZsq{}}
\end{Verbatim}


\begin{Verbatim}[commandchars=\\\{\}]
{\color{outcolor}Out[{\color{outcolor}17}]:} 'This is also a string'
\end{Verbatim}
            
    \begin{Verbatim}[commandchars=\\\{\}]
{\color{incolor}In [{\color{incolor}18}]:} \PY{l+s+sd}{\PYZdq{}\PYZdq{}\PYZdq{}This is a single string too\PYZdq{}\PYZdq{}\PYZdq{}}
\end{Verbatim}


\begin{Verbatim}[commandchars=\\\{\}]
{\color{outcolor}Out[{\color{outcolor}18}]:} 'This is a single string too'
\end{Verbatim}
            
    \begin{Verbatim}[commandchars=\\\{\}]
{\color{incolor}In [{\color{incolor}19}]:} \PY{l+s+sd}{\PYZsq{}\PYZsq{}\PYZsq{}This is not a string. Jokes, this totally is a string!\PYZsq{}\PYZsq{}\PYZsq{}}
\end{Verbatim}


\begin{Verbatim}[commandchars=\\\{\}]
{\color{outcolor}Out[{\color{outcolor}19}]:} 'This is not a string. Jokes, this totally is a string!'
\end{Verbatim}
            
    Strings can also be empty:

    \begin{Verbatim}[commandchars=\\\{\}]
{\color{incolor}In [{\color{incolor}20}]:} \PY{l+s+s2}{\PYZdq{}}\PY{l+s+s2}{\PYZdq{}}
\end{Verbatim}


\begin{Verbatim}[commandchars=\\\{\}]
{\color{outcolor}Out[{\color{outcolor}20}]:} ''
\end{Verbatim}
            
    You can combine strings using the \texttt{+} operator:

    \begin{Verbatim}[commandchars=\\\{\}]
{\color{incolor}In [{\color{incolor}21}]:} \PY{l+s+s2}{\PYZdq{}}\PY{l+s+s2}{Strings can be}\PY{l+s+s2}{\PYZdq{}} \PY{o}{+} \PY{l+s+s2}{\PYZdq{}}\PY{l+s+s2}{ combined!}\PY{l+s+s2}{\PYZdq{}}
\end{Verbatim}


\begin{Verbatim}[commandchars=\\\{\}]
{\color{outcolor}Out[{\color{outcolor}21}]:} 'Strings can be combined!'
\end{Verbatim}
            
    And you can even use the \texttt{*} operator to multiply a string!

    \begin{Verbatim}[commandchars=\\\{\}]
{\color{incolor}In [{\color{incolor}22}]:} \PY{l+s+s2}{\PYZdq{}}\PY{l+s+s2}{Cool,}\PY{l+s+s2}{\PYZdq{}} \PY{o}{+} \PY{l+m+mi}{3} \PY{o}{*} \PY{l+s+s2}{\PYZdq{}}\PY{l+s+s2}{ cool}\PY{l+s+s2}{\PYZdq{}}
\end{Verbatim}


\begin{Verbatim}[commandchars=\\\{\}]
{\color{outcolor}Out[{\color{outcolor}22}]:} 'Cool, cool cool cool'
\end{Verbatim}
            
    \subsubsection{Introducing print}\label{introducing-print}

At this point, it might be good to learn about the \texttt{print}
function. We'll get to functions n more detail later, but this one is
good to known about in the context of strings. When running a script,
output is not usually produced \emph{unless you specifically request
it}. The \texttt{print} function is a way to make this happen.

For example, if you assign a value to variable \texttt{ghost} (either in
a console or in a Jupyter Notebook), you don't get any visible feedback:

    \begin{Verbatim}[commandchars=\\\{\}]
{\color{incolor}In [{\color{incolor}23}]:} \PY{n}{ghost} \PY{o}{=} \PY{l+s+s2}{\PYZdq{}}\PY{l+s+s2}{man wearing a sheet}\PY{l+s+s2}{\PYZdq{}}
\end{Verbatim}


    Let's use print to shine some light on what \texttt{ghost} is:

    \begin{Verbatim}[commandchars=\\\{\}]
{\color{incolor}In [{\color{incolor}24}]:} \PY{k}{print}\PY{p}{(}\PY{n}{ghost}\PY{p}{)}
\end{Verbatim}


    \begin{Verbatim}[commandchars=\\\{\}]
man wearing a sheet

    \end{Verbatim}

    The \texttt{print} function can be used on any variable, not just on
strings. It can be useful in scripts to check the value of a particular
variable, or to give you periodic updates of what is happening.

    \subsubsection{String formatting}\label{string-formatting}

Strings come with their own formatting options. Specifically, you can
use \texttt{\{\}} as a placeholder for a value that you might want to
add later. An example:

    \begin{Verbatim}[commandchars=\\\{\}]
{\color{incolor}In [{\color{incolor}25}]:} \PY{n}{s} \PY{o}{=} \PY{l+s+s2}{\PYZdq{}}\PY{l+s+s2}{I am \PYZob{}\PYZcb{} years old}\PY{l+s+s2}{\PYZdq{}}
\end{Verbatim}


    To assign a value to the wildcard, you can use the sting's
\texttt{format} method:

    \begin{Verbatim}[commandchars=\\\{\}]
{\color{incolor}In [{\color{incolor}26}]:} \PY{k}{print}\PY{p}{(}\PY{n}{s}\PY{p}{)}
         \PY{k}{print}\PY{p}{(}\PY{n}{s}\PY{o}{.}\PY{n}{format}\PY{p}{(}\PY{l+m+mi}{42}\PY{p}{)}\PY{p}{)}
         \PY{k}{print}\PY{p}{(}\PY{n}{s}\PY{o}{.}\PY{n}{format}\PY{p}{(}\PY{l+s+s2}{\PYZdq{}}\PY{l+s+s2}{twenty}\PY{l+s+s2}{\PYZdq{}}\PY{p}{)}\PY{p}{)}
         \PY{k}{print}\PY{p}{(}\PY{n}{s}\PY{o}{.}\PY{n}{format}\PY{p}{(}\PY{l+m+mf}{2.5}\PY{p}{)}\PY{p}{)}
\end{Verbatim}


    \begin{Verbatim}[commandchars=\\\{\}]
I am \{\} years old
I am 42 years old
I am twenty years old
I am 2.5 years old

    \end{Verbatim}

    \subsection{Collections}\label{collections}

So far you've only seen single values, whether numbers or texts.
(Although strictly speaking a string could be considered a collection of
single characters!) There are also variable types that allow you to
combine those.

A \textbf{tuple} is a collection of values:

    \begin{Verbatim}[commandchars=\\\{\}]
{\color{incolor}In [{\color{incolor}27}]:} \PY{n}{a} \PY{o}{=} \PY{p}{(}\PY{l+m+mi}{1}\PY{p}{,} \PY{l+m+mi}{2}\PY{p}{,} \PY{l+m+mi}{3}\PY{p}{)}
\end{Verbatim}


    You can get at specific values in a list by using their \textbf{index}.
For this, Python starts counting at 0. So, it you would like to get at
the first value, you would use \texttt{a{[}0{]}}. The second value is at
\texttt{a{[}1{]}}, and so on:

    \begin{Verbatim}[commandchars=\\\{\}]
{\color{incolor}In [{\color{incolor}28}]:} \PY{k}{print}\PY{p}{(}\PY{n}{a}\PY{p}{[}\PY{l+m+mi}{0}\PY{p}{]}\PY{p}{)}
         \PY{k}{print}\PY{p}{(}\PY{n}{a}\PY{p}{[}\PY{l+m+mi}{1}\PY{p}{]}\PY{p}{)}
         \PY{k}{print}\PY{p}{(}\PY{n}{a}\PY{p}{[}\PY{l+m+mi}{2}\PY{p}{]}\PY{p}{)}
\end{Verbatim}


    \begin{Verbatim}[commandchars=\\\{\}]
1
2
3

    \end{Verbatim}

    Tuples cannot be altered (they can be completely overwriten)

    \begin{Verbatim}[commandchars=\\\{\}]
{\color{incolor}In [{\color{incolor}29}]:} \PY{n}{a}\PY{p}{[}\PY{l+m+mi}{0}\PY{p}{]} \PY{o}{=} \PY{l+m+mi}{2}
\end{Verbatim}


    \begin{Verbatim}[commandchars=\\\{\}]

        ---------------------------------------------------------------------------

        TypeError                                 Traceback (most recent call last)

        <ipython-input-29-fa1ba3f7cc8f> in <module>()
    ----> 1 a[0] = 2
    

        TypeError: 'tuple' object does not support item assignment

    \end{Verbatim}

    An alternative that CAN be altered is the \textbf{list}:

    \begin{Verbatim}[commandchars=\\\{\}]
{\color{incolor}In [{\color{incolor}31}]:} \PY{n}{a} \PY{o}{=} \PY{p}{[}\PY{l+m+mi}{1}\PY{p}{,} \PY{l+m+mi}{2}\PY{p}{,} \PY{l+m+mi}{3}\PY{p}{]}
\end{Verbatim}


    You can index lists in the same way as tuples:

    \begin{Verbatim}[commandchars=\\\{\}]
{\color{incolor}In [{\color{incolor}32}]:} \PY{k}{print}\PY{p}{(}\PY{n}{a}\PY{p}{[}\PY{l+m+mi}{0}\PY{p}{]}\PY{p}{)}
         \PY{k}{print}\PY{p}{(}\PY{n}{a}\PY{p}{[}\PY{l+m+mi}{1}\PY{p}{]}\PY{p}{)}
         \PY{k}{print}\PY{p}{(}\PY{n}{a}\PY{p}{[}\PY{l+m+mi}{2}\PY{p}{]}\PY{p}{)}
\end{Verbatim}


    \begin{Verbatim}[commandchars=\\\{\}]
1
2
3

    \end{Verbatim}

    You can alter any value in a list directly:

    \begin{Verbatim}[commandchars=\\\{\}]
{\color{incolor}In [{\color{incolor}33}]:} \PY{n}{a}\PY{p}{[}\PY{l+m+mi}{0}\PY{p}{]} \PY{o}{=} \PY{n}{a}\PY{p}{[}\PY{l+m+mi}{0}\PY{p}{]} \PY{o}{+} \PY{l+m+mi}{5}
         \PY{k}{print}\PY{p}{(}\PY{n}{a}\PY{p}{)}
\end{Verbatim}


    \begin{Verbatim}[commandchars=\\\{\}]
[6, 2, 3]

    \end{Verbatim}

    Altering the entire list at once is NOT possible:

    \begin{Verbatim}[commandchars=\\\{\}]
{\color{incolor}In [{\color{incolor}34}]:} \PY{n}{a} \PY{o}{=} \PY{n}{a} \PY{o}{+} \PY{l+m+mi}{5}
\end{Verbatim}


    \begin{Verbatim}[commandchars=\\\{\}]

        ---------------------------------------------------------------------------

        TypeError                                 Traceback (most recent call last)

        <ipython-input-34-460e7a5471e1> in <module>()
    ----> 1 a = a + 5
    

        TypeError: can only concatenate list (not "int") to list

    \end{Verbatim}

    \subsection{For loops}\label{for-loops}

If you do want to change all values in a list, you can do so one by one.
(In the example below, the \texttt{+=} operator is used. This means "add
to this variable". That means that \texttt{a\ =\ a\ +\ 5} is the same as
\texttt{a\ +=\ 5})

    \begin{Verbatim}[commandchars=\\\{\}]
{\color{incolor}In [{\color{incolor}35}]:} \PY{n}{a} \PY{o}{=} \PY{p}{[}\PY{l+m+mi}{1}\PY{p}{,} \PY{l+m+mi}{2}\PY{p}{,} \PY{l+m+mi}{3}\PY{p}{]}
         
         \PY{n}{a}\PY{p}{[}\PY{l+m+mi}{0}\PY{p}{]} \PY{o}{+}\PY{o}{=} \PY{l+m+mi}{5}
         \PY{n}{a}\PY{p}{[}\PY{l+m+mi}{1}\PY{p}{]} \PY{o}{+}\PY{o}{=} \PY{l+m+mi}{5}
         \PY{n}{a}\PY{p}{[}\PY{l+m+mi}{2}\PY{p}{]} \PY{o}{+}\PY{o}{=} \PY{l+m+mi}{5}
\end{Verbatim}


    For three values, this is still doable. But if your list has hundreds of
values, having to type hundreds of additional lines would take FOR EVER!
This is where loops come in!

You can use a \textbf{for loop} to cycle through all values in a
collection:

    \begin{Verbatim}[commandchars=\\\{\}]
{\color{incolor}In [{\color{incolor}36}]:} \PY{k}{for} \PY{n}{value} \PY{o+ow}{in} \PY{n}{a}\PY{p}{:}
             \PY{k}{print}\PY{p}{(}\PY{n}{value}\PY{p}{)}
\end{Verbatim}


    \begin{Verbatim}[commandchars=\\\{\}]
6
7
8

    \end{Verbatim}

    To be able to change the entire \texttt{a} list, you would need to be
able to get all indices. There is a function that can do this! It's
called \texttt{range}, and it provides a list with values from a
starting point and an ending point. Here, the starting point should be 0
(the first index), and the ending point should be the length of the
list. The length of the list can be computed with the \texttt{len}
function.

    \begin{Verbatim}[commandchars=\\\{\}]
{\color{incolor}In [{\color{incolor}37}]:} \PY{n}{start} \PY{o}{=} \PY{l+m+mi}{0}
         \PY{n}{end} \PY{o}{=} \PY{n+nb}{len}\PY{p}{(}\PY{n}{a}\PY{p}{)}
         \PY{n}{indices} \PY{o}{=} \PY{n+nb}{range}\PY{p}{(}\PY{n}{start}\PY{p}{,} \PY{n}{end}\PY{p}{)}
         \PY{k}{print}\PY{p}{(}\PY{n}{indices}\PY{p}{)}
\end{Verbatim}


    \begin{Verbatim}[commandchars=\\\{\}]
[0, 1, 2]

    \end{Verbatim}

    Now that you know how to get the indices, you can use them to index all
values in \texttt{a} in a single for loop:

    \begin{Verbatim}[commandchars=\\\{\}]
{\color{incolor}In [{\color{incolor}38}]:} \PY{k}{for} \PY{n}{i} \PY{o+ow}{in} \PY{n+nb}{range}\PY{p}{(}\PY{l+m+mi}{0}\PY{p}{,} \PY{n+nb}{len}\PY{p}{(}\PY{n}{a}\PY{p}{)}\PY{p}{)}\PY{p}{:}
             \PY{k}{print}\PY{p}{(}\PY{n}{a}\PY{p}{[}\PY{n}{i}\PY{p}{]}\PY{p}{)}
\end{Verbatim}


    \begin{Verbatim}[commandchars=\\\{\}]
6
7
8

    \end{Verbatim}

    Instead of printing all values in \texttt{a}, our plan was to change all
values in \texttt{a}. So let's do that with our newfound ability to loop
through all values in the list:

    \begin{Verbatim}[commandchars=\\\{\}]
{\color{incolor}In [{\color{incolor}39}]:} \PY{k}{print}\PY{p}{(}\PY{n}{a}\PY{p}{)}
         \PY{k}{for} \PY{n}{i} \PY{o+ow}{in} \PY{n+nb}{range}\PY{p}{(}\PY{l+m+mi}{0}\PY{p}{,} \PY{n+nb}{len}\PY{p}{(}\PY{n}{a}\PY{p}{)}\PY{p}{)}\PY{p}{:}
             \PY{n}{a}\PY{p}{[}\PY{n}{i}\PY{p}{]} \PY{o}{+}\PY{o}{=} \PY{l+m+mi}{5}
         \PY{k}{print}\PY{p}{(}\PY{n}{a}\PY{p}{)}
\end{Verbatim}


    \begin{Verbatim}[commandchars=\\\{\}]
[6, 7, 8]
[11, 12, 13]

    \end{Verbatim}

    \subsection{Booleans and logic}\label{booleans-and-logic}

A \emph{Boolean} is a variable that can only be one of two things:
\texttt{True} or \texttt{False}, which is equivalent to \texttt{1} or
\texttt{0}.

    \begin{Verbatim}[commandchars=\\\{\}]
{\color{incolor}In [{\color{incolor}40}]:} \PY{n}{a} \PY{o}{=} \PY{n+nb+bp}{True}
         \PY{n}{b} \PY{o}{=} \PY{n+nb+bp}{False}
\end{Verbatim}


    The fact that Booleans are really \texttt{0} and \texttt{1} can lead to
odd behaviour, for example when you multiple True with any number:

    \begin{Verbatim}[commandchars=\\\{\}]
{\color{incolor}In [{\color{incolor}41}]:} \PY{k}{print}\PY{p}{(}\PY{l+m+mi}{3} \PY{o}{*} \PY{n+nb+bp}{True}\PY{p}{)}
         \PY{k}{print}\PY{p}{(}\PY{o}{\PYZhy{}}\PY{l+m+mi}{1} \PY{o}{*} \PY{n+nb+bp}{True}\PY{p}{)}
\end{Verbatim}


    \begin{Verbatim}[commandchars=\\\{\}]
3
-1

    \end{Verbatim}

    If you want to combine Booleans in a more sensible way, you can use
\emph{logical operators}. These include \texttt{and}, \texttt{or}, and
\texttt{not}.

The \texttt{and} operator only returns \texttt{True} if ALL conditions
are \texttt{True}:

    \begin{Verbatim}[commandchars=\\\{\}]
{\color{incolor}In [{\color{incolor}42}]:} \PY{k}{print}\PY{p}{(}\PY{n+nb+bp}{True} \PY{o+ow}{and} \PY{n+nb+bp}{True}\PY{p}{)}
         \PY{k}{print}\PY{p}{(}\PY{n+nb+bp}{True} \PY{o+ow}{and} \PY{n+nb+bp}{False}\PY{p}{)}
         \PY{k}{print}\PY{p}{(}\PY{n+nb+bp}{False} \PY{o+ow}{and} \PY{n+nb+bp}{False}\PY{p}{)}
\end{Verbatim}


    \begin{Verbatim}[commandchars=\\\{\}]
True
False
False

    \end{Verbatim}

    The \texttt{or} operator returns \texttt{True} if ANY conditions are
`True:

    \begin{Verbatim}[commandchars=\\\{\}]
{\color{incolor}In [{\color{incolor}43}]:} \PY{k}{print}\PY{p}{(}\PY{n+nb+bp}{True} \PY{o+ow}{or} \PY{n+nb+bp}{True}\PY{p}{)}
         \PY{k}{print}\PY{p}{(}\PY{n+nb+bp}{True} \PY{o+ow}{or} \PY{n+nb+bp}{False}\PY{p}{)}
         \PY{k}{print}\PY{p}{(}\PY{n+nb+bp}{False} \PY{o+ow}{or} \PY{n+nb+bp}{False}\PY{p}{)}
\end{Verbatim}


    \begin{Verbatim}[commandchars=\\\{\}]
True
True
False

    \end{Verbatim}

    The \texttt{not} operator can reverse a Boolean:

    \begin{Verbatim}[commandchars=\\\{\}]
{\color{incolor}In [{\color{incolor}44}]:} \PY{k}{print}\PY{p}{(}\PY{o+ow}{not} \PY{n+nb+bp}{True}\PY{p}{)}
         \PY{k}{print}\PY{p}{(}\PY{o+ow}{not} \PY{n+nb+bp}{False}\PY{p}{)}
\end{Verbatim}


    \begin{Verbatim}[commandchars=\\\{\}]
False
True

    \end{Verbatim}

    \subsubsection{Comparisons}\label{comparisons}

Booleans can be the result of comparisons. You can check whether any two
variables are the same by using the \texttt{==} operator:

    \begin{Verbatim}[commandchars=\\\{\}]
{\color{incolor}In [{\color{incolor}45}]:} \PY{k}{print}\PY{p}{(}\PY{l+m+mi}{3} \PY{o}{==} \PY{l+m+mi}{5}\PY{p}{)}
         \PY{k}{print}\PY{p}{(}\PY{l+m+mi}{10} \PY{o}{==} \PY{l+m+mi}{10}\PY{p}{)}
         \PY{k}{print}\PY{p}{(}\PY{l+s+s2}{\PYZdq{}}\PY{l+s+s2}{a}\PY{l+s+s2}{\PYZdq{}} \PY{o}{==} \PY{l+s+s2}{\PYZdq{}}\PY{l+s+s2}{b}\PY{l+s+s2}{\PYZdq{}}\PY{p}{)}
\end{Verbatim}


    \begin{Verbatim}[commandchars=\\\{\}]
False
True
False

    \end{Verbatim}

    Other comparisons are "smaller than" (\texttt{\textless{}}), "greater
than" (\texttt{\textgreater{}}), "smaller than or equal to"
(\texttt{\textless{}=}), "greater than or equal to"
(\texttt{\textgreater{}=}), and "does not equal" (\texttt{!=}).

    \begin{Verbatim}[commandchars=\\\{\}]
{\color{incolor}In [{\color{incolor}46}]:} \PY{k}{print}\PY{p}{(}\PY{l+m+mi}{2} \PY{o}{\PYZlt{}} \PY{l+m+mi}{3}\PY{p}{)}
         \PY{k}{print}\PY{p}{(}\PY{l+s+s2}{\PYZdq{}}\PY{l+s+s2}{a}\PY{l+s+s2}{\PYZdq{}} \PY{o}{!=} \PY{l+s+s2}{\PYZdq{}}\PY{l+s+s2}{b}\PY{l+s+s2}{\PYZdq{}}\PY{p}{)}
         \PY{k}{print}\PY{p}{(}\PY{l+m+mi}{2} \PY{o}{+} \PY{l+m+mi}{3} \PY{o}{\PYZlt{}}\PY{o}{=} \PY{l+m+mi}{8} \PY{o}{\PYZhy{}} \PY{l+m+mi}{3}\PY{p}{)}
\end{Verbatim}


    \begin{Verbatim}[commandchars=\\\{\}]
True
True
True

    \end{Verbatim}

    The \texttt{and} operator can also be written as \texttt{\&}, and the
\texttt{or} operator can be written as \texttt{\textbar{}}:

    \begin{Verbatim}[commandchars=\\\{\}]
{\color{incolor}In [{\color{incolor}47}]:} \PY{k}{print}\PY{p}{(}\PY{n+nb+bp}{True} \PY{o+ow}{and} \PY{n+nb+bp}{False}\PY{p}{)}
         \PY{k}{print}\PY{p}{(}\PY{n+nb+bp}{True} \PY{o}{\PYZam{}} \PY{n+nb+bp}{False}\PY{p}{)}
         \PY{k}{print}\PY{p}{(}\PY{n+nb+bp}{True} \PY{o+ow}{or} \PY{n+nb+bp}{False}\PY{p}{)}
         \PY{k}{print}\PY{p}{(}\PY{n+nb+bp}{True} \PY{o}{|} \PY{n+nb+bp}{False}\PY{p}{)}
\end{Verbatim}


    \begin{Verbatim}[commandchars=\\\{\}]
False
False
True
True

    \end{Verbatim}

    \subsubsection{If statements}\label{if-statements}

Booleans can be used in logic statements. One of these is the
\texttt{if} statement, which can be used to run specific lines of code
depending on a situation. For example, you might want to check if a
value is even. An even value is divisible by two without a remainder.

    \begin{Verbatim}[commandchars=\\\{\}]
{\color{incolor}In [{\color{incolor}48}]:} \PY{n}{a} \PY{o}{=} \PY{l+m+mi}{4}
         
         \PY{k}{if} \PY{n}{a} \PY{o}{\PYZpc{}} \PY{l+m+mi}{2} \PY{o}{==} \PY{l+m+mi}{0}\PY{p}{:}
             \PY{k}{print}\PY{p}{(}\PY{l+s+s2}{\PYZdq{}}\PY{l+s+s2}{This value is even!}\PY{l+s+s2}{\PYZdq{}}\PY{p}{)}
\end{Verbatim}


    \begin{Verbatim}[commandchars=\\\{\}]
This value is even!

    \end{Verbatim}

    You can add alternative options to your if statement using \texttt{elif}
(short for "else if"). In this case, you could check if a value is odd,
meaning it will leave a remainder of 1 after being divided by two.

    \begin{Verbatim}[commandchars=\\\{\}]
{\color{incolor}In [{\color{incolor}49}]:} \PY{n}{a} \PY{o}{=} \PY{l+m+mi}{5}
         
         \PY{k}{if} \PY{n}{a} \PY{o}{\PYZpc{}} \PY{l+m+mi}{2} \PY{o}{==} \PY{l+m+mi}{0}\PY{p}{:}
             \PY{k}{print}\PY{p}{(}\PY{l+s+s2}{\PYZdq{}}\PY{l+s+s2}{This value is even!}\PY{l+s+s2}{\PYZdq{}}\PY{p}{)}
         \PY{k}{elif} \PY{n}{a} \PY{o}{\PYZpc{}} \PY{l+m+mi}{2} \PY{o}{==} \PY{l+m+mi}{1}\PY{p}{:}
             \PY{k}{print}\PY{p}{(}\PY{l+s+s2}{\PYZdq{}}\PY{l+s+s2}{This value is odd!}\PY{l+s+s2}{\PYZdq{}}\PY{p}{)}
\end{Verbatim}


    \begin{Verbatim}[commandchars=\\\{\}]
This value is odd!

    \end{Verbatim}

    In this case, the if statement covers all options, provided the variable
\texttt{a} points to an integer. In cases where there are more options,
you can add as many \texttt{elif} statements as you would like.

In addition to \texttt{if} and \texttt{elif}, there is a "catch-all"
type of statement too. You can use this to cover any options that hadn't
yet been covered in your \texttt{if}s and \texttt{elif}s:

    \begin{Verbatim}[commandchars=\\\{\}]
{\color{incolor}In [{\color{incolor}50}]:} \PY{n}{a} \PY{o}{=} \PY{l+s+s2}{\PYZdq{}}\PY{l+s+s2}{a}\PY{l+s+s2}{\PYZdq{}}
         
         \PY{k}{if} \PY{n}{a} \PY{o}{\PYZlt{}} \PY{l+m+mi}{0}\PY{p}{:}
             \PY{k}{print}\PY{p}{(}\PY{l+s+s2}{\PYZdq{}}\PY{l+s+s2}{This value is negative.}\PY{l+s+s2}{\PYZdq{}}\PY{p}{)}
         \PY{k}{elif} \PY{n}{a} \PY{o}{\PYZgt{}} \PY{l+m+mi}{0}\PY{p}{:}
             \PY{k}{print}\PY{p}{(}\PY{l+s+s2}{\PYZdq{}}\PY{l+s+s2}{This value is positive.}\PY{l+s+s2}{\PYZdq{}}\PY{p}{)}
         \PY{k}{else}\PY{p}{:}
             \PY{k}{print}\PY{p}{(}\PY{l+s+s2}{\PYZdq{}}\PY{l+s+s2}{This value must be 0!}\PY{l+s+s2}{\PYZdq{}}\PY{p}{)}
\end{Verbatim}


    \begin{Verbatim}[commandchars=\\\{\}]
This value is positive.

    \end{Verbatim}

    The above statement first checks if a value is below zero, then if it's
above 0, and finally concludes that if the value was neither below nor
above zero, it must be zero!

You might have noticed that it is assumed that \texttt{a} points to a
numerical value. You could try to make it a string or a Boolean, and see
what happens. Seriously, try it! You might find that a string is
misrecognised as a positive value, that \texttt{True} is also a positive
value, and that \texttt{False} is zero.

One way to avoid this mess, is to check whether \texttt{a} is actually a
number. Checking a variable's type can be done with the \texttt{type}
function:

    \begin{Verbatim}[commandchars=\\\{\}]
{\color{incolor}In [{\color{incolor}51}]:} \PY{n}{a} \PY{o}{=} \PY{l+m+mi}{5}
         
         \PY{k}{if} \PY{n+nb}{type}\PY{p}{(}\PY{n}{a}\PY{p}{)} \PY{o}{==} \PY{n+nb}{int} \PY{o+ow}{or} \PY{n+nb}{type}\PY{p}{(}\PY{n}{a}\PY{p}{)} \PY{o}{==} \PY{n+nb}{float}\PY{p}{:}
             \PY{k}{print}\PY{p}{(}\PY{l+s+s2}{\PYZdq{}}\PY{l+s+s2}{This is a number!}\PY{l+s+s2}{\PYZdq{}}\PY{p}{)}
         \PY{k}{else}\PY{p}{:}
             \PY{k}{print}\PY{p}{(}\PY{l+s+s2}{\PYZdq{}}\PY{l+s+s2}{This is not a number...}\PY{l+s+s2}{\PYZdq{}}\PY{p}{)}
\end{Verbatim}


    \begin{Verbatim}[commandchars=\\\{\}]
This is a number!

    \end{Verbatim}

    \subsubsection{Nesting and indentation}\label{nesting-and-indentation}

As you might have noticed, the if statements are followed by lines of
code that are \textbf{indented}: They are four spaces ahead of the
previous line. These four spaces are how Python know what should be run
by the if statement.

You can use this to \textbf{nest} if statements: Write one statement
within another. Let's try this by combining the previous examples:

    \begin{Verbatim}[commandchars=\\\{\}]
{\color{incolor}In [{\color{incolor}52}]:} \PY{n}{a} \PY{o}{=} \PY{l+m+mi}{5}
         
         \PY{k}{if} \PY{n+nb}{type}\PY{p}{(}\PY{n}{a}\PY{p}{)} \PY{o}{==} \PY{n+nb}{int} \PY{o+ow}{or} \PY{n+nb}{type}\PY{p}{(}\PY{n}{a}\PY{p}{)} \PY{o}{==} \PY{n+nb}{float}\PY{p}{:}
             \PY{k}{print}\PY{p}{(}\PY{l+s+s2}{\PYZdq{}}\PY{l+s+s2}{This is a number!}\PY{l+s+s2}{\PYZdq{}}\PY{p}{)}
             \PY{k}{if} \PY{n}{a} \PY{o}{\PYZlt{}} \PY{l+m+mi}{0}\PY{p}{:}
                 \PY{k}{print}\PY{p}{(}\PY{l+s+s2}{\PYZdq{}}\PY{l+s+s2}{This value is negative.}\PY{l+s+s2}{\PYZdq{}}\PY{p}{)}
             \PY{k}{elif} \PY{n}{a} \PY{o}{\PYZgt{}} \PY{l+m+mi}{0}\PY{p}{:}
                 \PY{k}{print}\PY{p}{(}\PY{l+s+s2}{\PYZdq{}}\PY{l+s+s2}{This value is positive.}\PY{l+s+s2}{\PYZdq{}}\PY{p}{)}
             \PY{k}{else}\PY{p}{:}
                 \PY{k}{print}\PY{p}{(}\PY{l+s+s2}{\PYZdq{}}\PY{l+s+s2}{This value must be 0!}\PY{l+s+s2}{\PYZdq{}}\PY{p}{)}
         \PY{k}{else}\PY{p}{:}
             \PY{k}{print}\PY{p}{(}\PY{l+s+s2}{\PYZdq{}}\PY{l+s+s2}{This is not a number...}\PY{l+s+s2}{\PYZdq{}}\PY{p}{)}
\end{Verbatim}


    \begin{Verbatim}[commandchars=\\\{\}]
This is a number!
This value is positive.

    \end{Verbatim}

    \subsection{Functions}\label{functions}

You already encountered four functions from basic Python:
\texttt{print}, \texttt{type}, \texttt{len}, and \texttt{range}. You can
also create your own! For example, you could take rewrite previous bit
of code as a function.

Function's require three things: a name (duh!), inputs, and outputs.
Here, your function name could be \texttt{pos\_or\_neg}, because the
code determines whether a value is positive, negative, or 0. The input
is a numerical value (\texttt{number}, and the output is a string
(\texttt{output}).

In Python, you can write this using \texttt{def}:

    \begin{Verbatim}[commandchars=\\\{\}]
{\color{incolor}In [{\color{incolor}53}]:} \PY{k}{def} \PY{n+nf}{pos\PYZus{}or\PYZus{}neg}\PY{p}{(}\PY{n}{number}\PY{p}{)}\PY{p}{:}
             \PY{k}{if} \PY{n}{number} \PY{o}{\PYZlt{}} \PY{l+m+mi}{0}\PY{p}{:}
                 \PY{n}{output} \PY{o}{=} \PY{l+s+s2}{\PYZdq{}}\PY{l+s+s2}{neg}\PY{l+s+s2}{\PYZdq{}}
             \PY{k}{elif} \PY{n}{number} \PY{o}{\PYZgt{}} \PY{l+m+mi}{0}\PY{p}{:}
                 \PY{n}{output} \PY{o}{=} \PY{l+s+s2}{\PYZdq{}}\PY{l+s+s2}{pos}\PY{l+s+s2}{\PYZdq{}}
             \PY{k}{else}\PY{p}{:}
                 \PY{n}{output} \PY{o}{=} \PY{l+s+s2}{\PYZdq{}}\PY{l+s+s2}{0}\PY{l+s+s2}{\PYZdq{}}
\end{Verbatim}


    We can now use this function:

    \begin{Verbatim}[commandchars=\\\{\}]
{\color{incolor}In [{\color{incolor}54}]:} \PY{k}{print}\PY{p}{(}\PY{n}{pos\PYZus{}or\PYZus{}neg}\PY{p}{(}\PY{l+m+mi}{5}\PY{p}{)}\PY{p}{)}
         \PY{k}{print}\PY{p}{(}\PY{n}{pos\PYZus{}or\PYZus{}neg}\PY{p}{(}\PY{o}{\PYZhy{}}\PY{l+m+mi}{10}\PY{p}{)}\PY{p}{)}
\end{Verbatim}


    \begin{Verbatim}[commandchars=\\\{\}]
None
None

    \end{Verbatim}

    OK, so you didn't actually see any output. Maybe it helps if you try
printing the \texttt{output} variable directly?

    \begin{Verbatim}[commandchars=\\\{\}]
{\color{incolor}In [{\color{incolor}55}]:} \PY{k}{print}\PY{p}{(}\PY{n}{output}\PY{p}{)}
\end{Verbatim}


    \begin{Verbatim}[commandchars=\\\{\}]

        ---------------------------------------------------------------------------

        NameError                                 Traceback (most recent call last)

        <ipython-input-55-a707e9187d7c> in <module>()
    ----> 1 print(output)
    

        NameError: name 'output' is not defined

    \end{Verbatim}

    Oh, no! An error!

The variable \texttt{output} is \textbf{local} to the function. It
exists within the function, but cannot be accessed outside of the
function. OOPS!

To make the output available outside of the function, it needs to be
\textbf{returned}:

    \begin{Verbatim}[commandchars=\\\{\}]
{\color{incolor}In [{\color{incolor}56}]:} \PY{k}{def} \PY{n+nf}{pos\PYZus{}or\PYZus{}neg}\PY{p}{(}\PY{n}{number}\PY{p}{)}\PY{p}{:}
             \PY{k}{if} \PY{n}{number} \PY{o}{\PYZlt{}} \PY{l+m+mi}{0}\PY{p}{:}
                 \PY{n}{output} \PY{o}{=} \PY{l+s+s2}{\PYZdq{}}\PY{l+s+s2}{neg}\PY{l+s+s2}{\PYZdq{}}
             \PY{k}{elif} \PY{n}{number} \PY{o}{\PYZgt{}} \PY{l+m+mi}{0}\PY{p}{:}
                 \PY{n}{output} \PY{o}{=} \PY{l+s+s2}{\PYZdq{}}\PY{l+s+s2}{pos}\PY{l+s+s2}{\PYZdq{}}
             \PY{k}{else}\PY{p}{:}
                 \PY{n}{output} \PY{o}{=} \PY{l+s+s2}{\PYZdq{}}\PY{l+s+s2}{0}\PY{l+s+s2}{\PYZdq{}}
             \PY{k}{return} \PY{n}{output}
\end{Verbatim}


    Let's try it now:

    \begin{Verbatim}[commandchars=\\\{\}]
{\color{incolor}In [{\color{incolor}57}]:} \PY{k}{print}\PY{p}{(}\PY{n}{pos\PYZus{}or\PYZus{}neg}\PY{p}{(}\PY{l+m+mi}{5}\PY{p}{)}\PY{p}{)}
         \PY{k}{print}\PY{p}{(}\PY{n}{pos\PYZus{}or\PYZus{}neg}\PY{p}{(}\PY{o}{\PYZhy{}}\PY{l+m+mi}{10}\PY{p}{)}\PY{p}{)}
         \PY{k}{print}\PY{p}{(}\PY{n}{pos\PYZus{}or\PYZus{}neg}\PY{p}{(}\PY{l+m+mi}{0}\PY{p}{)}\PY{p}{)}
\end{Verbatim}


    \begin{Verbatim}[commandchars=\\\{\}]
pos
neg
0

    \end{Verbatim}

    We currently assume that the \texttt{number} variable passed by whoever
uses the function is in fact a number. However, you can just as easily
pass something else, and end up in trouble:

    \begin{Verbatim}[commandchars=\\\{\}]
{\color{incolor}In [{\color{incolor}58}]:} \PY{k}{print}\PY{p}{(}\PY{n}{pos\PYZus{}or\PYZus{}neg}\PY{p}{(}\PY{l+s+s2}{\PYZdq{}}\PY{l+s+s2}{minus twelve}\PY{l+s+s2}{\PYZdq{}}\PY{p}{)}\PY{p}{)}
         \PY{k}{print}\PY{p}{(}\PY{n}{pos\PYZus{}or\PYZus{}neg}\PY{p}{(}\PY{l+s+s2}{\PYZdq{}}\PY{l+s+s2}{0}\PY{l+s+s2}{\PYZdq{}}\PY{p}{)}\PY{p}{)}
\end{Verbatim}


    \begin{Verbatim}[commandchars=\\\{\}]
pos
pos

    \end{Verbatim}

    The same solution as before applies here too: You need to check whether
the passed input value is actually a number. Only if it is a number, you
can return something sensible. If it is not a number, you can pass
\texttt{None}. This is a special variable type that basically is
nothing.

    \begin{Verbatim}[commandchars=\\\{\}]
{\color{incolor}In [{\color{incolor}59}]:} \PY{k}{def} \PY{n+nf}{pos\PYZus{}or\PYZus{}neg}\PY{p}{(}\PY{n}{number}\PY{p}{)}\PY{p}{:}
             \PY{k}{if} \PY{n+nb}{type}\PY{p}{(}\PY{n}{number}\PY{p}{)} \PY{o+ow}{is} \PY{n+nb}{int} \PY{o+ow}{or} \PY{n+nb}{type}\PY{p}{(}\PY{n}{number}\PY{p}{)} \PY{o+ow}{is} \PY{n+nb}{float}\PY{p}{:}
                 \PY{k}{if} \PY{n}{number} \PY{o}{\PYZlt{}} \PY{l+m+mi}{0}\PY{p}{:}
                     \PY{n}{output} \PY{o}{=} \PY{l+s+s2}{\PYZdq{}}\PY{l+s+s2}{neg}\PY{l+s+s2}{\PYZdq{}}
                 \PY{k}{elif} \PY{n}{number} \PY{o}{\PYZgt{}} \PY{l+m+mi}{0}\PY{p}{:}
                     \PY{n}{output} \PY{o}{=} \PY{l+s+s2}{\PYZdq{}}\PY{l+s+s2}{pos}\PY{l+s+s2}{\PYZdq{}}
                 \PY{k}{else}\PY{p}{:}
                     \PY{n}{output} \PY{o}{=} \PY{l+s+s2}{\PYZdq{}}\PY{l+s+s2}{0}\PY{l+s+s2}{\PYZdq{}}
             \PY{k}{else}\PY{p}{:}
                 \PY{n}{output} \PY{o}{=} \PY{n+nb+bp}{None}
             \PY{k}{return} \PY{n}{output}
\end{Verbatim}


    Let's try!

    \begin{Verbatim}[commandchars=\\\{\}]
{\color{incolor}In [{\color{incolor}60}]:} \PY{k}{print}\PY{p}{(}\PY{n}{pos\PYZus{}or\PYZus{}neg}\PY{p}{(}\PY{l+m+mi}{5}\PY{p}{)}\PY{p}{)}
         \PY{k}{print}\PY{p}{(}\PY{n}{pos\PYZus{}or\PYZus{}neg}\PY{p}{(}\PY{l+m+mi}{0}\PY{p}{)}\PY{p}{)}
         \PY{k}{print}\PY{p}{(}\PY{n}{pos\PYZus{}or\PYZus{}neg}\PY{p}{(}\PY{o}{\PYZhy{}}\PY{l+m+mi}{10}\PY{p}{)}\PY{p}{)}
         \PY{k}{print}\PY{p}{(}\PY{n}{pos\PYZus{}or\PYZus{}neg}\PY{p}{(}\PY{l+s+s2}{\PYZdq{}}\PY{l+s+s2}{5}\PY{l+s+s2}{\PYZdq{}}\PY{p}{)}\PY{p}{)}
         \PY{k}{print}\PY{p}{(}\PY{n}{pos\PYZus{}or\PYZus{}neg}\PY{p}{(}\PY{n+nb+bp}{True}\PY{p}{)}\PY{p}{)}
         \PY{k}{print}\PY{p}{(}\PY{n}{pos\PYZus{}or\PYZus{}neg}\PY{p}{(}\PY{n+nb+bp}{None}\PY{p}{)}\PY{p}{)}
\end{Verbatim}


    \begin{Verbatim}[commandchars=\\\{\}]
pos
0
neg
None
None
None

    \end{Verbatim}

    \subsection{Comments}\label{comments}

So far, you've only seen code that is runable. You could read it, but
the main purpose was for the computer to read it. In addition to this,
there is also code that is \emph{not} intended for the computer. This is
text that you add to your scripts specifically for humans who read the
code. These can be notes to yourself, or to anyone else who might have
to work with your code.

Comments are super useful. You might not appreciate this right now, but
human memory is fallible. Code that you read or wrote and understand
now, is going to be completely unintelligible in a few weeks time.
(Sometimes I don't even understand what I did before lunch...) Comments
allow you to explain what each part of your code does.

Let's add some comments to the function you wrote above. Comments are
preceded by the pound sign (\texttt{\#}), and Python will ignore them:

    \begin{Verbatim}[commandchars=\\\{\}]
{\color{incolor}In [{\color{incolor}61}]:} \PY{c+c1}{\PYZsh{} Function definitions start with `def`, and require}
         \PY{c+c1}{\PYZsh{} that the input is specified between brackets. Here,}
         \PY{c+c1}{\PYZsh{} `number` is the input.}
         \PY{k}{def} \PY{n+nf}{pos\PYZus{}or\PYZus{}neg}\PY{p}{(}\PY{n}{number}\PY{p}{)}\PY{p}{:}
             \PY{c+c1}{\PYZsh{} We can\PYZsq{}t assume that the passed value for `number`}
             \PY{c+c1}{\PYZsh{} is in fact a number. Whether it is should thus be}
             \PY{c+c1}{\PYZsh{} tested. The `type` function returns the type of}
             \PY{c+c1}{\PYZsh{} the passed variable. Numbers can be of type `int`}
             \PY{c+c1}{\PYZsh{} or `float`.}
             \PY{k}{if} \PY{n+nb}{type}\PY{p}{(}\PY{n}{number}\PY{p}{)} \PY{o+ow}{is} \PY{n+nb}{int} \PY{o+ow}{or} \PY{n+nb}{type}\PY{p}{(}\PY{n}{number}\PY{p}{)} \PY{o+ow}{is} \PY{n+nb}{float}\PY{p}{:}
                 \PY{c+c1}{\PYZsh{} All numbers below 0 are negative.}
                 \PY{k}{if} \PY{n}{number} \PY{o}{\PYZlt{}} \PY{l+m+mi}{0}\PY{p}{:}
                     \PY{n}{output} \PY{o}{=} \PY{l+s+s2}{\PYZdq{}}\PY{l+s+s2}{neg}\PY{l+s+s2}{\PYZdq{}}
                 \PY{c+c1}{\PYZsh{} All numbers above zero are positive.}
                 \PY{k}{elif} \PY{n}{number} \PY{o}{\PYZgt{}} \PY{l+m+mi}{0}\PY{p}{:}
                     \PY{n}{output} \PY{o}{=} \PY{l+s+s2}{\PYZdq{}}\PY{l+s+s2}{pos}\PY{l+s+s2}{\PYZdq{}}
                 \PY{c+c1}{\PYZsh{} This else statement should only be run if the}
                 \PY{c+c1}{\PYZsh{} value was not below 0, but also not above 0.}
                 \PY{c+c1}{\PYZsh{} The only number for which this is true is 0.}
                 \PY{k}{else}\PY{p}{:}
                     \PY{n}{output} \PY{o}{=} \PY{l+s+s2}{\PYZdq{}}\PY{l+s+s2}{0}\PY{l+s+s2}{\PYZdq{}}
             \PY{c+c1}{\PYZsh{} This else statement is run if the first if}
             \PY{c+c1}{\PYZsh{} statement isn\PYZsq{}t True. This only hapens if the}
             \PY{c+c1}{\PYZsh{} passed value for `number` was not an int or a}
             \PY{c+c1}{\PYZsh{} float. In these cases, `None` should be returned.}
             \PY{k}{else}\PY{p}{:}
                 \PY{n}{output} \PY{o}{=} \PY{n+nb+bp}{None}
             \PY{c+c1}{\PYZsh{} The return statement makes sure that the output is}
             \PY{c+c1}{\PYZsh{} available outside of this function. This means}
             \PY{c+c1}{\PYZsh{} that the function can be called like this: }
             \PY{c+c1}{\PYZsh{} `out = pos\PYZus{}or\PYZus{}neg(5)`}
             \PY{k}{return} \PY{n}{output}
\end{Verbatim}


    There is no such thing as \emph{too} elaborate commenting. One general
guideline is that the ratio of comments to code should be about 2:1.
That means two-thirds of your script would be non-functional comments.
(It is very rare to actually see this in practice, but it's much less
rare to curse code that is under-commented!)

    \section{Modules that extend Python}\label{modules-that-extend-python}

All of the above is available in Python without having to do anything.
However, there are also functions that are tucked away in modules. Basic
Python comes with a lot of these, each adding specific functionality.
This includes, for example, the \texttt{math} module for mathematical
operations (think \texttt{sin}, \texttt{cos}, \texttt{tan}, etc.), the
\texttt{os} module for things relating to the operating system,
\texttt{multiprocessing} and \texttt{threading} for parallel computing
needs, and MANY more.

These modules offer quite specific functionality, so there's a good
chance you won't need them. To avoid any confusion or accidental
overwriting of function/variable names, the additional modules are thus
not loaded by default. Instead, you need to load them explicitly.

For example, say that you wanted to use the number \(\pi\) in your
script. It's a part of the \texttt{math} module, which you have to load
before using it:

    \begin{Verbatim}[commandchars=\\\{\}]
{\color{incolor}In [{\color{incolor}62}]:} \PY{k+kn}{import} \PY{n+nn}{math}
         
         \PY{k}{print}\PY{p}{(}\PY{n}{math}\PY{o}{.}\PY{n}{pi}\PY{p}{)}
\end{Verbatim}


    \begin{Verbatim}[commandchars=\\\{\}]
3.14159265359

    \end{Verbatim}

    \subsection{NumPy}\label{numpy}

In addition to the built-in modules, there exist \emph{many} external
packages that offer even more functionality. One of the most important
ones is \textbf{NumPy}, short for numeric Python. This package is a
crucial tool for many people, and basically unmissable to scientists.

One of the cool elements of NumPy is the \texttt{array}. This is a
variable type that can represent vectors or matrices of values, usually
numerical value. (But they can also be strings, or any other variable
type.) One of the main advantages of the NumPy array is that you can use
arithmetic operations on the whole array. This is unlike the
\texttt{list}, which had to be looped through. Side-by-side comparison:

    \begin{Verbatim}[commandchars=\\\{\}]
{\color{incolor}In [{\color{incolor}63}]:} \PY{k+kn}{import} \PY{n+nn}{numpy}
         
         \PY{c+c1}{\PYZsh{} Change all values in a list:}
         \PY{n}{l} \PY{o}{=} \PY{p}{[}\PY{l+m+mi}{1}\PY{p}{,} \PY{l+m+mi}{2}\PY{p}{,} \PY{l+m+mi}{3}\PY{p}{]}
         \PY{k}{for} \PY{n}{i} \PY{o+ow}{in} \PY{n+nb}{range}\PY{p}{(}\PY{n+nb}{len}\PY{p}{(}\PY{n}{l}\PY{p}{)}\PY{p}{)}\PY{p}{:}
             \PY{n}{l}\PY{p}{[}\PY{n}{i}\PY{p}{]} \PY{o}{+}\PY{o}{=} \PY{l+m+mi}{5}
         \PY{k}{print}\PY{p}{(}\PY{n}{l}\PY{p}{)}
         
         \PY{c+c1}{\PYZsh{} Change all values in a NumPy array:}
         \PY{n}{a} \PY{o}{=} \PY{n}{numpy}\PY{o}{.}\PY{n}{array}\PY{p}{(}\PY{p}{[}\PY{l+m+mi}{1}\PY{p}{,}\PY{l+m+mi}{2}\PY{p}{,}\PY{l+m+mi}{3}\PY{p}{]}\PY{p}{)}
         \PY{n}{a} \PY{o}{+}\PY{o}{=} \PY{l+m+mi}{5}
         \PY{k}{print}\PY{p}{(}\PY{n}{a}\PY{p}{)}
\end{Verbatim}


    \begin{Verbatim}[commandchars=\\\{\}]
[6, 7, 8]
[6 7 8]

    \end{Verbatim}

    Arithmetic isn't the only thing you can do with a whole array. You can
also compare all values in the whole array at the same time against a
single value, or you can compare all values in one array against all
values in an array of the same shape:

    \begin{Verbatim}[commandchars=\\\{\}]
{\color{incolor}In [{\color{incolor}64}]:} \PY{n}{a} \PY{o}{=} \PY{n}{numpy}\PY{o}{.}\PY{n}{array}\PY{p}{(}\PY{p}{[}\PY{l+m+mi}{1}\PY{p}{,} \PY{l+m+mi}{2}\PY{p}{,} \PY{l+m+mi}{3}\PY{p}{,} \PY{l+m+mi}{4}\PY{p}{]}\PY{p}{)}
         \PY{n}{b} \PY{o}{=} \PY{n}{numpy}\PY{o}{.}\PY{n}{array}\PY{p}{(}\PY{p}{[}\PY{l+m+mi}{0}\PY{p}{,} \PY{l+m+mi}{2}\PY{p}{,} \PY{l+m+mi}{0}\PY{p}{,} \PY{l+m+mi}{4}\PY{p}{]}\PY{p}{)}
         
         \PY{k}{print}\PY{p}{(}\PY{n}{a} \PY{o}{==} \PY{l+m+mi}{2}\PY{p}{)}
         \PY{k}{print}\PY{p}{(}\PY{n}{a} \PY{o}{==} \PY{n}{b}\PY{p}{)}
\end{Verbatim}


    \begin{Verbatim}[commandchars=\\\{\}]
[False  True False False]
[False  True False  True]

    \end{Verbatim}

    The results of these comparisons are \textbf{Boolean arrays}. You can
combine these in the same way as single Booleans:

    \begin{Verbatim}[commandchars=\\\{\}]
{\color{incolor}In [{\color{incolor}65}]:} \PY{n}{a} \PY{o}{=} \PY{n}{numpy}\PY{o}{.}\PY{n}{array}\PY{p}{(}\PY{p}{[}\PY{n+nb+bp}{True}\PY{p}{,} \PY{n+nb+bp}{False}\PY{p}{,} \PY{n+nb+bp}{True}\PY{p}{,} \PY{n+nb+bp}{False}\PY{p}{]}\PY{p}{)}
         \PY{n}{b} \PY{o}{=} \PY{n}{numpy}\PY{o}{.}\PY{n}{array}\PY{p}{(}\PY{p}{[}\PY{n+nb+bp}{True}\PY{p}{,} \PY{n+nb+bp}{True}\PY{p}{,} \PY{n+nb+bp}{False}\PY{p}{,} \PY{n+nb+bp}{False}\PY{p}{]}\PY{p}{)}
         
         \PY{k}{print}\PY{p}{(}\PY{n}{a} \PY{o}{|} \PY{n}{b}\PY{p}{)}
         \PY{k}{print}\PY{p}{(}\PY{n}{a} \PY{o}{\PYZam{}} \PY{n}{b}\PY{p}{)}
\end{Verbatim}


    \begin{Verbatim}[commandchars=\\\{\}]
[ True  True  True False]
[ True False False False]

    \end{Verbatim}

    NumPy is full of useful functions, and you will be seeing a lot more of
it today!


    % Add a bibliography block to the postdoc
    
    
    
    \end{document}
